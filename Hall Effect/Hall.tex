\documentclass{article}
\usepackage{blindtext}
\usepackage[utf8]{inputenc}
 
\title{Hall Effect in Plasmas}
\author{Kevin Chen}
\date{\today}
 
\begin{document}
\maketitle

\section*{HAL Pre Lab and Evaluation}
\subsection*{Pre-lab Discussion Questions}
\begin{enumerate}
\item The Hall Effect is when charges, in the presence of a magnetic field, experience the Lorentz Force and move towards one part of a conductor. As charges accumulate on one side of the conductor, a voltage difference is created between the two sides of the conductor, creating an electric field that opposes further migration of charges?caused by the magnetic field? so long as the charges moving across the conductor. \\
We use a plasma in this experiment, instead of a metal, because it the Hall Effect is very large and easy to observe.
\item To say that the plasma has a temperature is to say that, in Kelvins, we can measure the thermal kinetic energy per particle using statistical mechanics. Even though the temperatures of plasmas are incredibly high (in the thousands of Kelvins), the density of the particles is low, making the actual energy of the entire plasma low. Therefore, because even though the energy per particle is high, the density of particles is low, so the energy transfer from the electrons the to helium gas is not very high, resulting in a low enough temperature not to melt the gas. 
\item For this experiment, we need to measure the following: The pressure of the plasma, the strength of the magnetic field, the Hall Voltage, the voltage longitudinally across the glass tube, and the longitudinal current density. Using these parameters we can determine the following: the momentum loss frequency $\nu_q$, the number density of the plasma $n_q$, the resistivity of the material that the plasma is traveling through $\eta \equiv m_q \nu_q / n_q q^2$, and the temperature of plasma $T_q$.
To find the electron density, we can use the relation  $\mathbf{E_H} \simeq - \mathbf{j}/q n_q \times \mathbf{B}$. To find $\nu_q$, we can use the two relations $\rho = E_x/j_x$ and $\rho = m\nu / q^2 n$.
\item 2500 Volts
\item We do not use the DVM to measure the relevant voltages because the DVM has too low of an impedance ($\approx 10^7 \Omega$), whereas the voltmeter provided for us in the lab has an impedance on the order of giga ohms $(\approx 10^9 \Omega$).

\end{enumerate}



\end{document}
