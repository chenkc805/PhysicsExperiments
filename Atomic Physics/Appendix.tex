%%%%%%%%%%%%%%%%%%%%%%%%%%%%%%%%%%%%%%%%%%%%%%%%%%%%%%%%%%%%%%%%%%%%%%%
% Sample template for MIT Junior Lab Student Written Summaries
% Available from http://web.mit.edu/8.13/samplepaper/sample-paper.tex
%
% Last Updated June 20, 2004
%
% Adapted from the American Physical Societies REVTeK-4 Pages
% at http://publish.aps.org
%
% ADVICE TO STUDENTS: Each time you write a paper, start with this
%    template and save under a new filename.  If convenient, don't
%    erase unneeded lines, just comment them out.  Often, they
%    will be useful containers for information.
%%%%%%%%%%%%%%%%%%%%%%%%%%%%%%%%%%%%%%%%%%%%%%%%%%%%%%%%%%%%%%%%%%%%%%%


%%%%%%%%%%%%%%%%%%%%%%%%%%%%%%%%%%%%%%%%%%%%%%%%%%%%%%%%%%%%%%%%%%%%%%%
% PREAMBLE
% The preamble of a LaTeX document is the set of commands that precede
% the \begin{document} line.  It contains a \documentclass line
% to load the REVTeK-4 macro definitions and various \usepackage
% lines to load other macro packages.
%
% ADVICE TO STUDENTS: This preamble contains a suggested set of
%     class options to generate a ``Junior Lab'' look and feel that
%     facilitate quick review and feedback from one's peers, TA's
%     and section instructors.  Don't make substantial changes without
%     first consulting your section instructor.
%%%%%%%%%%%%%%%%%%%%%%%%%%%%%%%%%%%%%%%%%%%%%%%%%%%%%%%%%%%%%%%%%%%%%%%

\documentclass[aps,twocolumn,secnumarabic,nobalancelastpage,amsmath,amssymb,
nofootinbib]{revtex4}

% nofootinbib is another document class option that allows you to put
% footnotes on the page where they occur rather than at the end of the
% paper.  This makes for easier reading!

% secnumarabic is a particularly nice way of identifying sections by
% number to aid electronic review and commentary.

% amsmath and amssymb are necessary for the subequations environment
% among others

\usepackage{graphics}      % standard graphics specifications
\usepackage{graphicx}      % alternative graphics specifications
\usepackage{longtable}     % helps with long table options
\usepackage{url}           % for on-line citations
\usepackage{bm}            % special 'bold-math' package
\usepackage{subfigure}
\usepackage{booktabs}

%%%%%%%%%%%%%%%%%%%%%%%%%%%%%%%%%%%%%%%
%                                 %%%%%
% And now, begin the document...  %%%%%
%                                 %%%%%
%%%%%%%%%%%%%%%%%%%%%%%%%%%%%%%%%%%%%%%
\begin{document}

\setcounter{page}{8}

\section*{Appendix}

\begin{figure}[hhh]
  \begin{center}
    \subfigure[Spectral Broadening in the Alpha Line. The fit is centered around 6668 (6667, 6669) with a standard deviation of 4.61599 (4.04889, 5.18309). The values in parentheses are the 95$\%$ confidence intervals.]{\includegraphics[width=7.5 cm]{alphaline.png}}
    \subfigure[Spectral Broadening in the Beta Line. The fit is centered around  4965 (4965, 4965) with a standard deviation of 4.72347 (4.3282, 5.11875).]{\includegraphics[width=7.5 cm]{betaline.png}}
    \subfigure[Spectral Broadening in the Gamma Line. The fit is centered around 4446 (4446, 4446) with a standard deviation of 4.93773 (4.77156, 5.1039).]{\includegraphics[width=7.5 cm]{gammaline.png}}
  \end{center}
  \caption{Spectral Broadening exhibited in Hydrogen Balmer Series lines.}
  \label{Appendix}
\end{figure}

%%%%%%%%%%%%%%%%%%%%%%%%%%%%%%%%%%%%%%%%%%%%%%%%%%%%%%%%%%%%%%%%%%%%%%%%%%%%%

\end{document}
